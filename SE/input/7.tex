\section*{\fontsize{16}{14}\selectfont Aim: Generate the test cases using boundary value analysis for some problems.}
\subsection*{Analysis}
In computer programming, testing is a software method by which individual units of source code, sets of one or more computer program modules together with associated control data, usage procedures, and operating procedures are tested to determine if they are fit for use. Intuitively, one can view a unit as the smallest testable part of an application. In procedural programming, a unit could be an entire module, but it is more commonly an individual function or procedure. In object-oriented programming, a unit is often an entire interface, such as a class, but could be an individual method. Unit tests are short code fragments created by programmers or occasionally by white box testers during the development process. Ideally, each test case is independent from the others. Substitutes such as method stubs, mock objects, fakes, and test harnesses can be used to assist testing a module in isolation. Unit tests are typically written and run by software developers to ensure that code meets its design and behaves as intended. 

\subsection*{Testing}
Testing a program consists of providing the program with a set of test inputs (or test cases) and
observing if the program behaves as expected. If the program fails to behave as expected, then the
conditions under which failure occurs are noted for later debugging and correction. \\
This software had been taken through rigorous test to fully found potential causes of error and system failure
and full focus have been given to cover all possible exceptions that can 
occur and cause failure of the software.\\
As this software is based on intensive background process it have been taken care that 
if correct input and email address are given then processing of user job can even continue or a least automatically 
restart even after server shuts down or even crash.
\begin{table}[h]
\centering
\begin{tabular}{ ||c|c|| }
\hline
 \multicolumn{2}{||c||}{Overview of Octave versions} \\
 \hline
 Date & Publication Title \\ [0.5ex] 
 \hline \hline
	September 1999 & Octave framework 1.0 \\ \hline
	September 2001 & Octave framework 2.0 \\ \hline
	December 2001 & Octave criteria 2.0\\ \hline
	September 2003 & Octave-S v0.9 \\ \hline
	March 2005 & Octave-S v1.0 \\ \hline
	June 2007 & Octave 3.x\\ \hline
        March 2016 &  Octave 4.0.1        \\ \hline
	July 2016 & Octave 4.0.3 \\ [1ex]
 \hline
\end{tabular}
\caption{Octave Release}
\label{table2}
\end{table}

\begin{table}[h]
\centering
\resizebox{\textwidth}{!}{\begin{tabular}{ |c|c|c|c| }
\hline
 \multicolumn{4}{|c|}{Test cases for main page(index.html) } \\[0.5ex]
 \hline
 Input & Desired Output & Actual Output & Status \\ [0.5ex] 
 \hline \hline
 Inputs range exceeds & Alert user,Don't proceed & Alert user,Don't proceed. & Pass\\ \hline
 Incomplete Command & Alert user about range. Don't proceed & Alert user about range. Don't proceed & Pass\\ \hline
 PNG selected: No & Error & Don't proceed & Pass\\ \hline
 Session: Yes & Show email field after Submit & Show email field after Submit & Pass\\ \hline
 Help pressed  & Show Detailed user help & Show Detailed user help  & Pass\\ \hline
 
\end{tabular}}
\caption{Computational Analysis}
\label{table3}
\end{table}
\begin{table}[h]
\centering
\resizebox{\textwidth}{!}{
\begin{tabular}{ |c|c|c|c| }
\hline
 \multicolumn{4}{|c|}{Test cases for possible source of problems } \\[0.5ex]
 \hline
  Input & Desired Output & Actual Output & Status \\ [0.5ex] 
 \hline \hline 
 URL refreshed & Send to homepage & Send to homepage & Pass\\ \hline
 server stops or rebooted & Start processing interrupted requests & Start processing interrupted requests & Pass\\ \hline
\end{tabular}}



\caption{Test case (general)}
\label{table}
\end{table}
