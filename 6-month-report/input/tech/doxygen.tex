\section{Introduction to Doxygen}
\begin{figure}[ht]
\centering \includegraphics[scale=1]{input/images/doxygen.jpeg}
\caption{Doxygen logo}
\end{figure}
\noindent Doxygen is a documentation generator, a tool for writing software reference 
documentation. The documentation is written within code, and is thus 
relatively easy to keep up to date. Doxygen can cross reference 
documentation and code, so that the reader of a document can easily 
refer to the actual code.

Doxygen supports multiple programming languages, especially C++, C, 
C\#, Objective-C, Java, Python, IDL, VHDL, Fortran and PHP.[2] Doxygen
 is free software, released under the terms of the GNU General Public 
License.\\

\subsection{Features of Doxygen}
Doxygen is a tool to create a documentation for your program/project written in the languages like C, C++, Java, python and so on. It reads the well formatted and special doxygen comments to create the required documentation. This documentation is very important for the new developers who want to help in the development of the project. Documentation is one of the main pillar of an open-source project.
\begin{itemize}
\item Requires very little overhead from the writer of the documentation. 
Plain text will do, Markdown is support, and for more fancy or structured 
output HTML tags and/or some of doxygen's special commands can be used.
\item Cross platform: Works on Windows and many Unix flavors (including 
Linux and Mac OS X).
\item Comes with a GUI frontend (Doxywizard) to ease editing the options 
and run doxygen. The GUI is available on Windows, Linux, and Mac OS X.
\item Automatically generates class and collaboration diagrams in HTML 
(as clickable image maps) and $\mbox{\LaTeX}$ (as Encapsulated PostScript 
images).
\item Allows grouping of entities in modules and creating a hierarchy 
of modules.
\item Doxygen can generate a layout which you can use and edit to change 
the layout of each page.
\item Can cope with large projects easily.
\end{itemize}
\subsection{Installation of Doxygen}
Doxygen can be installed using following commands:\\\\
\emph{
\$ git clone https://github.com/doxygen/doxygen.git\\\\
\$ cd doxygen\\\\
\$ ./configure\\\\
\$ make \\\\}
This will install Doxygen on your pc or laptop.\\
You can create the documentation using the graphical user interface (GUI) or console mode.\\\\
While writing the comments we have to follow a pattern with the tags i.e. before every tag we should have something special so that Doxygen can understand what are we creating. Actually Doxygen read these tags and place them at special location in the generated output. So, we have to specify them explicitly. 



