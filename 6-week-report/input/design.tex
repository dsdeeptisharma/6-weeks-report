
\section{Problem Formulation}
 When analytical solution of the mathematically defined problem is possible but it is time-consuming and the error of approximation we can use Octave. In this case the calculations are mostly made with use of computer because otherwise its highly doubtful if any time is saved. It is indivually to decide what do we mean by "time-consuming analytical solution". In my discipline even very simple mechanical problems are solved numerically simply because of laziness. \\
\noindent When analytical solution is impossible, which was discussed by eg. Alexander Sadovsky. This means that we have to apply numerical methods in order to find the solution. This does not define that we must do calculations with computer although it usually happens so because of the number of required operations.

\section{Facilities required for proposed work}
\subsection{Hardware Requirements}
\begin{itemize}
\item Operating System: Linux/Windows
\item Processor Speed: 512KHz or more
\item RAM: Minimum 256MB
\end{itemize}
\subsection{Software Requirements}
\begin{itemize}
\item Software: octave gui
\item Programming Language: octave
\end{itemize}

\section{Methodology}
\begin{itemize}
\item Studying various methods available to solve different problems of numerical analysis.
\item Deciding various input and output parameters of methods.
\item Making the approach modular.
\item Graphical representation of solutions wherever possible.
\item Generating documentation.
\end{itemize}

\section{Project Work} 
\textbf{Studied Previous System:}\\
Before starting the project. \\\\
\textbf{Learn octave:}\\
Before starting with project, we have to go through the basics of Octave, such that there
should not be any problem proceeding with project.\\\\
\textbf{Get Familiar with Different methods and their algorithms:}\\
We have gone through algorithms of these algorithms. Then implementation becomes easy.\\\\
\textbf{Functions:}\\
The user has been provided some test functions which he can use to test various.\\\\
\textbf{Plots:}\\
Octave provides fltk as the default toolkit. But we can use gnuplot for more accurate plotting by setting them as default toolkit.\\\\
\textbf{Input:}\\
Input values are taken from user or default values defined in the file are used.\\\\
\textbf{Output:}\\
The iterations are performed and it returns the output with the expected precision.

