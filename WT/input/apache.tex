
\section*{\fontsize{16}{14}\selectfont Aim: Configuration and administration of IIS and Apache Web Server.}
\subsection*{\underline{\textbf{Manual Installation of Apache Server}}}
Manual installation offers several benefits: 
\begin{itemize}
\item Backing up, reinstalling, or moving the web server can be achieved in seconds.
\item You have more control over how and when Apache starts .
\item You can install Apache anywhere, such as a portable USB drive (useful for client demonstrations). 
\end{itemize}

\subsection*{Step 1: Configure IIS, Skype and other software (optional)}

If you have a Professional or Server version of Windows, you may already have IIS installed. If you would prefer Apache, either remove IIS as a Windows component or disable its services.
Apache listens for requests on TCP/IP port 80. The default installation of Skype also listens on this port and will cause conflicts. To switch it off, start Skype and choose Tools > Options > Advanced > Connection. Ensure you untick "Use port 80 and 443 as alternatives for incoming connections".
\subsection*{Step 2: Download the files}
We are going to use the unofficial Windows binary from Apache Lounge. This version has performance and stability improvements over the official Apache distribution, although I am yet to notice a significant difference. However, it is provided as a manually installable ZIP file from www.apachelounge.com/download/
You should also download and install the Windows C++ runtime from Microsoft.com. You may have this installed already, but there is no harm installing it again.
As always, remember to virus scan all downloads.
\subsection*{Step 3: Extract the files}
We will install Apache in C:Apache2, so extract the ZIP file to the root of the C: drive.
Apache can be installed anywhere on your system, but you will need to change the configuration file paths accordingly.
\begin{figure}[!ht]
\centering
\includegraphics[width=0.6\textwidth]{/home/deepti/Desktop/2.png}                   
\caption{Installation of Apache}
\hspace{-1.5em}
\end{figure}

\subsection*{Step 4: Configure Apache}
Apache is configured with the text file confhttpd.conf contained in the Apache folder. Open it with your favourite text editor.
Note that all file path settings use a forward-slash rather than the Windows backslash. If you installed Apache anywhere other than C:Apache2, now is a good time to search and replace all references to "C:/Apache2".
There are several lines you should change for your production environment. Some of them are:\\
Line 46, listen to all requests on port 80:\\
Listen *:80\\
Line 172, specify the server domain name:\\
Line 224, allow .htaccess overrides:\\
AllowOverride All
\subsection*{Step 5: Change the web page root (optional)}
By default, Apache return files found in its htdocs folder. I would recommend using a folder on an another drive or partition to make backups and re-installation easier. For the purposes of this example, we will create a folder called D:WebPages and change httpd.conf accordingly:\\
Line 179, set the root:\\
DocumentRoot "D:/WebPages"\\
and line 204:\\
<Directory "D:/WebPages">
\subsection*{Step 6: Test your Installation}
Your Apache configuration can now be tested. Open a command box (Start > Run > cmd) and enter:\\
cd Apache2bin\\
httpd -t\\
Correct any httpd.conf configuration errors and retest until none appear.
\subsection*{Step 7: Install Apache as Windows Service}
The easiest way to start Apache is to add it as a Windows service. From a command prompt, enter:\\
cd Apache2bin\\
httpd -k install\\
Open the Control Panel, Administrative Tools, then Services and double-click Apache2.2. Set the Startup type to "Automatic" to ensure Apache starts every time you boot your PC.\\
Alternatively, set the Startup type to "Manual" and launch Apache whenever you choose using the command "net start Apache2.2".
\subsection*{Step 8: Test the Web Server }
Create a file named index.html in Apache's web page root (either htdocs or D:WebPages) and add a little HTML code:\\
<html>\\
<head><title>testing Apache</title></head>\\
<body><p>Apache is working!</p></body>\\
</html>\\
Ensure Apache has started successfully, open a web browser and enter the address http://localhost/. If all goes well, your test page should appear.\\
In general, most problems will be caused by an incorrect setting in the httpd.conf configuration file. Refer to the Apache documentation if you require further information.

